% -*- tex -*-
%
% Summary:
%
% This is the preface text. It is included by the "main.tex", and
% cannot my be compiled by itself (since it lacks the neccessary
% "\begin{document}...\end{document} for this).
%
%%%%%%%%%%%%%%%
%%  Preface  %%
%%%%%%%%%%%%%%%

\parindent=1em                                  %% en fyrkants indrag
\lhead[\large\B{Source Abbreviations}]{}%       %% Left header col
\rhead[]{\large\B{Introduction \& Word Type Abbreviations}}% Right head col

\raggedbottom


\header{Introduction}

\noindent This dictionary consists of a collection of the Klingon
words that could be found in the various works of the inventor of the
language, Dr.\ Marc Okrand. Since our intention has been to produce a
practical pocket reference guide, rather than a complete description
of the language, it is assumed that you have at least basic knowledge
of the structure and grammar of Klingon. If you do not, we urge you to
read Okrand's main book on the subject, entitled \I{The Klingon
Dictionary} (Pocket Books, New York, 1992, ISBN~0-671-74559-X).

The contents of this book was automatically created from a database
containing \fromkliwords{} Klingon words and \tokliwords{} English
lookup entries. The database was created in late 1997 and has since
been continuously updated. Should you find any error or omission, or
if you would like to know more about \I{Kling\-on\-ska Aka\-demi\-en}
and our various projects, please contact us at:

\vspace{3mm}%
\noindent\parbox[t]{.5\textwidth}{%
  \rule{1cm}{0mm}\I{Klingonska Akademien}\\
  \rule{1cm}{0mm}Villavägen 33, 2tr\\
  \rule{1cm}{0mm}SE--752~36~Uppsala, Sweden%
}%
\parbox[t]{.5\textwidth}{%
  \rule{1cm}{0mm}\I{http://klingonska.org/}\\
  \rule{1cm}{0mm}\I{paqhom@klingonska.org}\\
  \rule{1cm}{0mm}+46~(0)76~211~50~43%
}


\header{Word Type Abbreviations}

\noindent This book uses the same abbreviations for word type as TKD, with the
addition of ``name'' which is used to indicate names of people. Only names
mentioned in okrandian sources are included in this dictionary.

\begin{center}\begin{tabular}{lll}
\I{v}    & verb          & {\small [TKD~4]}   \\
\I{n}    & noun          & {\small [TKD~3]}   \\
\I{name} & name          & {\small [TKD~5.6]} \\
\I{pro}  & pronoun       & {\small [TKD~5.1]} \\
\I{adv}  & adverb        & {\small [TKD~5.4]} \\
\I{num}  & number        & {\small [TKD~5.2]} \\
\I{excl} & exclamation   & {\small [TKD~5.5]} \\
\I{ques} & question word & {\small [TKD~6.4]} \\
\I{conj} & conjunction   & {\small [TKD~5.3]}
\end{tabular}\end{center}

%s:v      |verb        
%s:s      |substantiv  
%s:namn   |namn        
%s:pro    |pronomen    
%s:adv    |adverb      
%s:räkn   |räkneord    
%s:interj |interjektion
%s:fråg   |frågeord    
%s:konj   |konjunktion 


\header{Source Abbreviations}

\noindent All Klingon words in this dictionary come from verifiable canon
sources, but those listed below are the most frequently occurring. However,
some sources are mentioned only rarely and are therefore not abbreviated. Most
notably, when a word originated from one of Okrand's many Usenet postings, the
source is given simply as ``News,'' with the date given
in \mbox{\I{YYYY-MM-DD}} format. To learn more about a specific source, take a
look at the \I{Archive of Okrand\-ian Canon}
(\I{http:/\xslash{}klingonska.org\xslash{}canon/}). Additional information can
also be found in \I{The Klingon Mailing List FAQ}
(\I{http:/\xslash{}higbee.cots.net\xslash{}Holtej\xslash{}klingon\xslash{}faq.htm}).

\vspace{7mm}
\begin{center}
\begin{tabular}{lll}
\B{BoP} & \I{Klingon Bird of Prey Cutaway Poster}\rule[-3mm]{0mm}{3mm}\\
\B{CK}  & \I{Conversational Klingon}  (audio recording)\rule[-3mm]{0mm}{3mm}\\
\B{HQ}  & \I{\B{HolQeD}}              (journal of the \I{Klingon Language Institute})\rule[-3mm]{0mm}{3mm}\\
\B{KCD} & \I{Star Trek: Klingon!}       (computer game \& language lab)\rule[-3mm]{0mm}{3mm}\\
\B{KGT} & \I{Klingon for the Galactic Traveler} (book)\rule[-3mm]{0mm}{3mm}\\
\B{KLI} & \I{The Klingon Language Institute}\rule[-3mm]{0mm}{3mm}\\
\B{MO}  & Marc Okrand                 (inventor of the language)\rule[-3mm]{0mm}{3mm}\\
\B{PK}  & \I{Power Klingon}           (audio recording)\rule[-3mm]{0mm}{3mm}\\
\B{S\#} & \I{SkyBox Trading Card S\#}\rule[-3mm]{0mm}{3mm}\\
\B{ST5} & \I{Star Trek V: The Final Frontier} (motion picture)\rule[-3mm]{0mm}{3mm}\\
\B{STE} & \I{Star Trek Encyclopedia} (book)\rule[-3mm]{0mm}{3mm}\\
 & (used only for English spelling)\rule[-3mm]{0mm}{3mm}\\
\B{TKD} & \I{The Klingon Dictionary} (book)\rule[-3mm]{0mm}{3mm}\\
\B{TKDa}& Addendum to \I{The Klingon Dictionary}\rule[-3mm]{0mm}{3mm}\\
\B{TKW} & \I{The Klingon Way} (book)\rule[-3mm]{0mm}{3mm}\\
\B{E-K} & Only found in the English-Klingon part of source\rule[-3mm]{0mm}{3mm}\\
\B{K-E} & Only found in the Klingon-English part of source\rule[-3mm]{0mm}{3mm}\\
\end{tabular}
\end{center}


\newpage

\lhead[\large\B{Alphabet and Pronunciation \& Stress}]{}% Left header col
\rhead[]{\large\B{Alphabet and Pronunciation}}% Right header col
%\chead[\large\B{Alphabet and Pronunciation}]%
%  {\large\B{Alphabet and Pronunciation}} % Remove page header

\header{Alphabet and Pronunciation}

\noindent The Klingon alphabetical order is as follows:

\begin{center}
\B{a}, \B{b}, \B{ch}, \B{D}, \B{e}, \B{gh}, \B{H}, \B{I}, \B{j},
\B{l}, \B{m}, \B{n}, \B{ng}, \\
\B{o}, \B{p}, \B{q}, \B{Q}, \B{r}, \B{S}, \B{t}, \B{tlh}, \B{u},
\B{v}, \B{w}, \B{y}, \B{'}
\end{center}

\noindent Note that \B{ch}, \B{gh}, \B{ng}, \B{tlh} and \B{'} are
considered letters in their own right, and that, as a result of this,
the word \B{nob} would come \I{before} \B{ngab} in a Klingon
alphabetic listing. \B{q} and \B{Q} represent two different sounds, and
are sorted separately.

This is only a rough guide to Klingon pronunciation, for a more
detailed description, see TKD section 1.

\vspace{3mm}

\phonexpl{a}%
  {\textipa{[A]} As in \I{ps\U{a}lm} or \I{p\U{a}}, never as in
  \I{cr\U{a}b\U{a}pple}.}{open back unrounded}

\phonexpl{b}%
  {\textipa{[b]} As in \I{\U{b}ronchitis}, \I{gaze\U{b}o} or
  \I{\U{b}ri\U{b}e}.}{voiced bilabial stop}

\phonexpl{ch}%
  {\textipa{[\t{tS}]} As in \I{\U{ch}ew} or
  \I{arti\U{ch}oke}.}{voiceless postalveolar affricate}

\phonexpl{D}%
  {\textipa{[\:d]} As in Swedish \I{vä\U{rd}} (host), further back
  than English \I{d} as in \I{\U{d}ream} or \I{an\U{d}roid}. Let the
  tongue touch halfway between the teeth and the soft palate.}{voiced
  retroflex stop}

\phonexpl{e}%
  {\textipa{[E]} As in \I{s\U{e}nsor} or \I{p\U{e}t}.}{open-mid front
  unrounded}

\phonexpl{gh}%
  {\textipa{[G]} Put tongue as if to say \I{gobble}, but relax and
  hum. Almost the same as \B{H} but voiced.}{voiced velar fricative}

\phonexpl{H}%
  {\textipa{[x]} As in the name of the german composer
  \I{Ba\U{ch}}. Very strong and coarse. Similar to \B{gh} but without
  humming.}{voiceless velar fricative}

\phonexpl{I}%
  {\textipa{[I]} As in \I{m\U{i}sf\U{i}t} or \I{p\U{i}t}.}{semi-close
  front unrounded}

\phonexpl{j}%
  {\textipa{[\t{dZ}]} As in \I{\U{j}unk} (with an initial
  \I{d}-sound), never as in French \I{jour}.}{voiced postalveolar
  affricate}

\phonexpl{l}%
  {\textipa{[l]} As in \I{\U{l}unge} or \I{a\U{l}chemy}.}{voiced
  alveolar lateral approximant}

\phonexpl{m}%
  {\textipa{[m]} As in \I{\U{m}ud} or \I{pneu\U{m}atic}.}{voiced
  bilabial nasal}

\phonexpl{n}%
  {\textipa{[n]} As in \I{\U{n}ectari\U{n}e} or \I{su\U{n}spot}.}{voiced
  alveolar nasal}

\phonexpl{ng}%
  {\textipa{[N]} As in \I{furlo\U{ng}} or \I{thi\U{ng}}, never as in
  \I{e\U{ng}ulf}. Also occurs at the beginning of syllables.}{voiced
  velar nasal}

\phonexpl{o}%
  {\textipa{[o]} As in \I{g\U{o}} or \I{m\U{o}saic}.}{close-mid back
  rounded}

\phonexpl{p}%
  {\textipa{[p\super h]} As in \I{\U{p}arallax} or \I{o\U{pp}robrium},
  always with a strong puff or pop, never laxly.}{aspirated voiceless
  bilabial stop}

\phonexpl{q}%
  {\textipa{[q\super h]} Similar to \I{k} in \I{kumquat}, but further
  back. The tongue should touch the uvula while saying this. A puff of
  air should accompany the sound.}{aspirated voiceless uvular stop}

\phonexpl{Q}%
  {\textipa{[\t{qX}]} A harder variant of \B{q}, very strong and
  raspy.}{voiceless uvular affricate}

\phonexpl{r}%
  {\textipa{[r]} or \textipa{[\*r]} A trilled \I{r} using the tip of
  the tongue, as in Swedish \I{\U{r}ö\U{r}} (pipe, tube) if properly
  articulated.}{voiced alveolar trill}

%\phonexpl{S}%
%  {\textipa{[\:s]} As in Swedish \I{fo\U{rs}} (rapid stream) or as an
%  English \I{s} articulated with the tongue in the Klingon \B{D}
%  position.}{voiceless retroflex fricative}

\phonexpl{S}%
  {\textipa{[\:s]} As in Swedish \I{mothå\U{rs}} (against the
  predominant direction of hair growth e.g. on a pet) or as an English
  \I{s} articulated with the tongue in the Klingon \B{D}
  position.}{voiceless retroflex fricative}

\phonexpl{t}%
  {\textipa{[t\super h]} As in \I{\U{t}arpaulin} or
  \I{cri\U{t}ique}. It is accompanied by a puff of air.}{aspirated
  voiceless dental stop}

\phonexpl{tlh}%
  {\textipa{[\t{t\textbeltl}]} To learn how to say this Klingon sound,
  first say \I{l}, then keep your tongue in the same position and
  exhale. Now repeat this, but let the air build up pressure behind
  your tongue before releasing it. The resulting sound should be
  voiceless, and you should be able to feel the air escape quite
  forcefully on both sides of your tongue.}{voiceless alveolar
  lateral affricate}

\phonexpl{u}%
  {\textipa{[u]} As in \I{gn\U{u}}, \I{pr\U{u}ne} or \I{s\U{oo}n},
  never as in \I{b\U{u}t} or \I{c\U{u}te}.}{close back rounded}

\phonexpl{v}%
  {\textipa{[v]} As in \I{\U{v}ulgar} or
  \I{demonstrati\U{v}e}.}{voiced labiodental fricative}

\phonexpl{w}%
  {\textipa{[w]} As in \I{\U{w}orry\U{w}art} or \I{co\U{w}}.}{voiced
  rounded labiovelar approximant}

\phonexpl{y}%
  {\textipa{[j]} As in \I{\U{y}odel} or \I{jo\U{y}}.}{voiced palatal
  central approximant}

\phonexpl{'}%
  {\textipa{[P]} As in the abrupt cut-off of sound in \I{uh-oh} or
  \mbox{\I{unh-unh}} meaning ``no''. At the end of a word this sound
  is usually followed by a soft echo of the preceding sound.}{glottal
  stop}


\header{Stress}

\noindent Verbs are always stressed on the last syllable of the word
itself, and the first suffix is never stressed. If, after that, there
are any suffixes which end in \B{'} then they are stressed, too. There
is an exception to the above: If a speaker wishes to emphasize a
particular suffix (as is often the case with the interrogative suffix
and suffixes for negation and emphasis) then stress may shift to that
suffix and leave the rest of the word unstressed. Also note that
adjectival verbs are stressed in the same way as other verbs.

Nouns are usually stressed on the last syllable of the stem. But if
there is a syllable in the word, or any of its suffixes, which ends in
a \B{'}, then that syllable is stressed instead. If there is more than
one such syllable, then they are all stressed. Also note that nouns
made from a verb plus \B{-wI'} or \B{-ghach} are stressed as nouns.

\clearpage
