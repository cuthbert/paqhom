% -*- tex -*-
%
%%%%%%%%%%%%%%%%%%%%%%%%%%%%%%%%%%%%%%%%%%%%%%%%
%                                              %
%        N E W   S T U F F   B E L O W         %
%                                              %
%%%%%%%%%%%%%%%%%%%%%%%%%%%%%%%%%%%%%%%%%%%%%%%%


\newcounter{pagecounter}
\newcommand{\doheader}{%
    \setcounter{pagecounter}{\value{firstpage}}%
    \ifthenelse{\value{page} = \value{pagecounter}}{%
      Verb Suffixes Type 1: Oneself/One Another%
    }{}%
    \addtocounter{pagecounter}{1}%
    \ifthenelse{\value{page} = \value{pagecounter}}{%
      Verb Suffixes Type 4: Cause%
    }{}%
    \addtocounter{pagecounter}{1}%
    \ifthenelse{\value{page} = \value{pagecounter}}{%
      Verb Suffixes Type 7: Aspect%
    }{}%
    \addtocounter{pagecounter}{1}%
    \ifthenelse{\value{page} = \value{pagecounter}}{%
      Verb Suffixes Type 9: Syntactic Markers%
    }{}%
    \addtocounter{pagecounter}{1}%
    \ifthenelse{\value{page} = \value{pagecounter}}{%
      Verb Suffixes Type 9: Syntactic Markers%
    }{}%
    \addtocounter{pagecounter}{1}%
    \ifthenelse{\value{page} = \value{pagecounter}}{%
      Verb Suffix Rovers%
    }{}%
    \addtocounter{pagecounter}{1}%
    \ifthenelse{\value{page} = \value{pagecounter}}{
      Noun Suffixes Type 4: Possession/Specification%
    }{}%
    \addtocounter{pagecounter}{1}%
    \ifthenelse{\value{page} = \value{pagecounter}}{
      Noun Suffixes Type 5: Syntactic Markers%
    }{}%
}




\newlength{\SuffixWidth}
\settowidth{\SuffixWidth}{\normalsize\B{-chugh }}
\newlength{\TypeNumberWidth}
\settowidth{\TypeNumberWidth}{\header{R. }}


% header for the suffix tables
%\newcommand{\chaptermark}{\markboth{\chaptername\ \thechapter. #1}}
\newlength{\SuffixGuideTypeRefRaise}
\newcommand{\SuffixGuideType}[3]{%
  % Created: 2001-12-18 by Zrajm
  % args: #1 = suffix type number e.g. "5"
  %       #2 = suffix type label  e.g. "Indefinite Subject/Ability"
  %       #3 = from TKD chapter   e.g. "TKD~4.2.5"
  %
  \par\noindent%
  \makebox[\linewidth]{%
    {%
      \noindent%
      \header{%
        \makebox[\TypeNumberWidth][l]{%
          #1.%
        }%
        #2%
        \hfill%
        \small [#3]%
      }%
    }%
  }\nopagebreak%
}

% subheader for the suffix tables
\newcommand{\SuffixGuideSubtype}[1]{%
  % Created: 2001-12-18 by Zrajm
  % args: #1 {suffix subtype label}            % e.g. "Subordinate-Clause Markers"
  \par\noindent%
  \makebox[\TypeNumberWidth]{}%                % left indent
  \parbox[t]{ \linewidth - \TypeNumberWidth }{%
    \B{#1}%
  }\nopagebreak%                                           %
}

% description of suffix(es)
\newcommand{\SuffixGuideDefinition}[5][TKD]{%
  % Created: 2001-12-18 by Zrajm
  % args: #1 [TKD reference] (optional)        % "TKDa" if found in addendum
  %       #2 {suffix}%                         % e.g. "'e'"
  %       #3 {suffix translation}%             % e.g. "Topic"
  %       #4 {description}%                    % e.g. "Emphasizes that a noun..."
  %       #5 {description reference}           % e.g. "TKDa 6.7"
  \par\noindent%
  \makebox[\TypeNumberWidth]{}%                % left indent
  \makebox[\SuffixWidth][l]{\B{-#2}}%          % suffix (in Klingon)
  \parbox[t]{\linewidth-\TypeNumberWidth-\SuffixWidth}{%
    \par\lowercase{\I{#3}}%                        % translation (in lowercase)
    \ifthenelse{\equal{#1}{TKD}}{}{{ {\small [#1]}}}% display if source = TKDa
    \ifthenelse{\equal{#4}{}}{}{%            % if there's a description
      \par #4%                                      % description
      \ifthenelse{\equal{#5}{}}{}{ {\small[#5]}}%% description reference
      \medskip%
    }%
  }%
}



%%%%%%%%%%%%%%%%%%%%%%%%%%%%%%%%%%%%%%%%%%%%%%%%
%                                              %
%        N E W   S T U F F   A B O V E         %
%                                              %
%%%%%%%%%%%%%%%%%%%%%%%%%%%%%%%%%%%%%%%%%%%%%%%%

\raggedbottom

\newcounter{firstpage}%                        % remember this page number
\setcounter{firstpage}{\value{page}}%          %
\lhead[\large\mbox{\textbf\doheader}]{}%
\rhead[]{\large\mbox{\textbf\doheader}}%


%%%%%%%%%%%%%%%%%%%%%%%%%%%%%%%%%%%%%%%%%%%%%%%%%%%%%%%%%%%%%%%%%%%%%%%%%%%%%%
%%%%%%%%%%%%%%%%%%%%%%%%%%%%%%%%%%%%%%%%%%%%%%%%%%%%%%%%%%%%%%%%%%%%%%%%%%%%%%
%%%%%%%%%%%%%%%%%%%%%%%%%%%%%%%%%%%%%%%%%%%%%%%%%%%%%%%%%%%%%%%%%%%%%%%%%%%%%%

\parindent=1em                                  %% en fyrkants indrag

\setlength{\fboxsep}{0cm}%                     % padding inside fbox
\setlength{\fboxrule}{.01mm}%                  % line thickness of boxes

\setlength{\arrayrulewidth}{.6mm}%             % line thickness in table
\setlength{\tabcolsep}{3\arrayrulewidth}%      % horizontal padding in cells
\setlength{\doublerulesep}{0mm}%               % distance between double lines
\renewcommand{\arraystretch}{.9}%              % table row height relative font
\setlength{\extrarowheight}{0mm}%              % extra space above text in each cell
\setlength{\minrowclearance}{2pt}

%\raggedbottom


\header{\LARGE Verb Suffixes}%
\vspace{3mm}

\SuffixGuideType{1}{Oneself/One Another}{TKD~4.2.1}%
  \SuffixGuideDefinition{'egh}{Oneself}{Indicates that the action
    affects the subject; requires a prefix indicating that there is no
    object. Can be used together with \B{-moH} to form a command of a
    stative verb, e.g.\ \B{yItuj'eghmoH} \I{Heat yourself!}}{KGT~117}
  \SuffixGuideDefinition{chuq}{One another, each other}{Requires a
    prefix indicating plural subject and no object.}{}
%
%
%\hline
\SuffixGuideType{2}{Volition/Predisposition}{TKD~4.2.2}
  \SuffixGuideDefinition{nIS}{Need}{}{}
  \SuffixGuideDefinition{qang}{Willing}{}{}
  \SuffixGuideDefinition{rup}{Ready, prepared (referring to beings)}{}{}
  \SuffixGuideDefinition{beH}{Ready, set up (referring to devices)}{}{}
  \SuffixGuideDefinition{vIp}{Afraid}{It is a cultural taboo to use
    the suffix \B{-vIp} with \I{I} or \I{we} as subject.}{}
%
%
%\hline
\SuffixGuideType{3}{Change}{TKD~4.2.3}
  \SuffixGuideDefinition{choH}{Change in state, change in
    direction}{E.g.\ \B{chomuSchoH} \I{I am beginning to hate you} (but
    I did not hate you before). The sentence \B{pa' ghoSchoH}
    \I{He/she is starting to go there} implies that either the person
    was not going anywhere before, or that he/she changed
    direction.}{}
  \SuffixGuideDefinition{qa'}{Resume, do again}{Indicates that the
      action stopped, then began again, e.g.\ \B{wInejqa'} \I{We are
      resuming searching for it} or \I{We search for it again.}}{}
%
%
%\hline
\SuffixGuideType{4}{Cause}{TKD~4.2.4}
  \SuffixGuideDefinition{moH}{Cause}{The subject causes a change in
    condition or creates a new one, e.g.\ \B{qul vIchenmoH} \I{I light
    a fire} (lit. \I{I cause a fire to take form}). Makes intransitive
    verbs transitive, e.g.\ \B{yIqIjmoH} \I{Blacken it!}
    (lit. \I{Cause it to be black!}). Required when making an
    imperative out of a stative verb (see also \B{-'egh} above).}{}
%
%
%\hline
\SuffixGuideType{5}{Indefinite Subject/Ability}{TKD~4.2.5}
  \SuffixGuideDefinition{lu'}{Indefinite subject}{Indicates that the
    subject is unknown, indefinite, and/or general, the verb can not
    have a subject, and the prefixes are used in a different way
    (see bottom row of prefix table). Sentences using \B{-lu'} are
    often translated into English passive voice, e.g.\ \B{Daqawlu'}
    \I{You are remembered.}}{}
  \SuffixGuideDefinition{laH}{Can, able}{E.g.\ \B{jIQonglaH} \I{I can
    sleep;} \B{tlhIngan Hol vIjatlhlaH} \I{I am able to speak
    Klingon.}}{}
%
%
%\hline
\SuffixGuideType{6}{Qualification}{TKD~4.2.6}
  \SuffixGuideDefinition{chu'}{Clearly, perfectly}{Indicates that an
    action is performed absolutely properly.}{PK}
  \SuffixGuideDefinition{bej}{Certainly, undoubtedly}{}{}
  \SuffixGuideDefinition[TKDa]{ba'}{Obviously}{Indicates that the
    speaker thinks what he/she says should be obvious to the listener,
    e.g.\ \B{QIpba'} \I{He/she is obviously stupid.} There is still
    room for doubt though, the suffix does not imply as strong a
    conviction as \B{-bej.}}{}
  \SuffixGuideDefinition{law'}{Seemingly, apparently}{Expresses that
    the speaker is uncertain, and may even be thought of as meaning
    \I{I think} or \I{I suspect,} e.g.\ \B{DuSeHlaw'} \I{He/she seems
    to be controlling you} or \I{I think he/she is controlling
    you.}}{}
%
%
%\hline
\SuffixGuideType{7}{Aspect}{TKD~4.2.7}
  \SuffixGuideDefinition{pu'}{Perfective}{Indicates that the action
    is completed.}{}
  \SuffixGuideDefinition{ta'}{Accomplished, done}{Indicates that the
    action was deliberately undertaken and completed.}{}
  \SuffixGuideDefinition{taH}{Continuous}{Indicates that the action
    is ongoing.}{}
  \SuffixGuideDefinition{lI'}{In progress}{Indicates that the action
    is ongoing and proceeding toward a known goal.}{}
%
%
%\hline
\SuffixGuideType{8}{Honorific}{TKD~4.2.8}
  \SuffixGuideDefinition{neS}{Honorific}{Indicates extreme politeness
    or deference. Used \I{only} when addressing a superior, e.g.\ 
    \B{HIja'neS} \I{Do me the honor of telling me.} It is never
    required.}{}
%
%
%\hline
\SuffixGuideType{9}{Syntactic Markers}{TKD~4.2.9}
%\SuffixGuideSpacer
  \SuffixGuideSubtype{Subordinate-Clause Markers}
  \SuffixGuideDefinition{DI'}{When, as soon as}{}{}
  \SuffixGuideDefinition{chugh}{If}{}{}
  \SuffixGuideDefinition{pa'}{Before}{}{}
  \SuffixGuideDefinition{vIS}{While}{}{}
  \SuffixGuideDefinition[TKDa]{mo'}{Due to, because of}{A subordinate
    clause can occur either before or after the rest of the sentence,
    e.g.\ \B{cha yIbaH qara'DI'} or \B{qara'DI' cha yIbaH} \I{Fire the
    torpedoes at my command!} The suffix \B{-vIS} is always used along
    with the type 7 suffix \B{-taH}, e.g.\ \B{bIQongtaHvIS} \I{while
    you are sleeping.} Note that there is also a noun suffix \B{-mo'}
    with the same meaning.}{TKD~6.2.2}
%
%\SuffixGuideSpacer
  \SuffixGuideSubtype{Relative-Clause Marker}
  \SuffixGuideDefinition{bogh}{Which}{A relative clause takes the
    place of noun in a sentence. The relative clause itself has a head
    noun, to which its verb refers, e.g.\ \B{qIp\-bogh yaS} \I{the
    officer who hit him/her} or \B{yaS qIp\-bogh} \I{the officer whom
    he/she hit.} If there is more than one noun in the clause, the
    head noun is indicated with the suffix \B{-'e'} \I{topic,}
    e.g.\ \B{loD\-Hom qIp\-bogh mang\-'e'} \I{the soldier who hit the
    boy.}}{TKD~6.2.3; TKW~p.189}
%
%\SuffixGuideSpacer
  \SuffixGuideSubtype{Purpose-Clause Marker}
  \SuffixGuideDefinition{meH}{For, for the purpose of, in order
    to}{The purpose clause always precedes the noun or verb whose
    purpose it is describing, e.g.\ \B{ja'\-chuq\-meH roj\-Hom} \I{a
    truce in order to confer} or \B{jagh lu\-HoH\-meH
    lu\-nej\-taH} \I{They are searching for the enemy in order to kill
    him/her.}}{TKD~6.2.4}
%
%\SuffixGuideSpacer
  \SuffixGuideSubtype{Main-Clause Modifiers}
  \SuffixGuideDefinition{'a'}{Interrogative}{Indicates that a sentence
    is a yes/no question, e.g.\ \B{bI\-jang\-'a'} \I{Will you
    answer?}}{TKD~6.4}
  \SuffixGuideDefinition[TKDa]{jaj}{may, let}{Expresses a desire or wish
    on the part of the speaker that something take place in the
    future. If used in a toast (but not otherwise) the sentence word
    order becomes object-subject-verb. E.g.\ \B{wo' ghaw\-ran Dev\-taH\-jaj}
    \I{May Gowron continue to lead the Empire,} if the same thing were
    to be expressed as a wish or aspiration on the speaker's part, and
    not a toast, it would be said \B{wo' Dev\-taH\-jaj ghaw\-ran}
    instead. Note: Klingons seem to be a bit touchy on the subject of
    toasts, and so it is important to use only the handful of accepted
    toasts.}{PK; \mbox{KGT~p.25--26}}
%
%\SuffixGuideSpacer
  \SuffixGuideSubtype{Nominalizers (Turns Verb into Noun)}
  \SuffixGuideDefinition{wI'}{One who does/is, thing which does/is}{In
    reference to inanimate objects it means \I{thing which does/is} or
    \I{thing which is used for,} when referring to beings it means
    \I{one who does/is.} E.g.\ \B{joq\-wI'} \I{flag;} \B{nan\-wI'}
    \I{chisel;} \B{baH\-wI'} \I{gunner;} \B{puj\-wI'} \I{weakling.} Also
    used to say things like \B{Doq\-wI'} \I{the red one.}}{TKD~3.2.2}
  \SuffixGuideDefinition[TKDa]{ghach}{Nominalizer}{Turns a verb (which must
    have at least one other suffix attached) into a noun. The use of this
    suffix often makes for bad Klingon, and it is strongly suggested that
    you refrain from using any word with \B{-ghach,} unless it is explicitly
    listed in the dictionary. E.g.\ \B{naD\-Ha'\-ghach} \I{discommendation;}
    \B{naD\-qa'\-ghach} \I{re-commendation.}}{}
%
%
%\hline
\SuffixGuideType{R}{Rovers}{TKD~4.3}
  \SuffixGuideDefinition{be'}{Not}{This suffix follows the element
    (verb or verb suffix) which it negates, e.g.\ \B{choHoHvIpbe'}
    \I{You are not afraid to kill me,} \B{choHoHbe'vIp} \I{You are
    afraid to not kill me.}  It can not be used in imperatives (where
    \B{-Qo'} is used instead), but it can be applied to verbs used
    adjectivally, e.g.\ \B{yIHmey lI'be'} \I{useless
    tribbles}}{TKDa~4.2.9; CK}
  \SuffixGuideDefinition{Qo'}{Don't!, won't!}{This suffix always
    occurs last, unless followed by a type 9 suffix. It is used in
    imperatives and to denote refusal.}{}
  \SuffixGuideDefinition{Ha'}{Undo}{Always occurs immediately after the
    verb, first among the suffixes. It indicates that something that
    was previously done is now undone, or that something is done
    wrongly, e.g.\ \B{nob\-Ha'} \I{give back;} \B{yaj\-Ha'}
    \I{misunderstand.} Can also be applied to verbs used
    adjectivally, e.g.\ \B{'ey\-Ha'} \I{undelicious;} \B{yep\-Ha'}
    \I{careless.}}{KGT~pp.30,~84,~150}
  \SuffixGuideDefinition{qu'}{Emphatic}{This suffix follows the
    element (verb or verb suffix) which it emphasizes,
    e.g.\ \B{nI\-muS\-law'\-qu'} \I{They SEEM to hate you,} \B{nI\-muS\-qu'\-law'}
    \I{They seem to HATE you.} Can also be applied to verbs used
    adjectivally, e.g.\ \B{veng tIn\-qu'} \I{very big city.}}{TKD~4.4}



\vfill\par\noindent%
\rule{\linewidth}{1pt}%
\par\header{\LARGE Noun Suffixes}%
\vspace{3mm}

\SuffixGuideType{1}{Size/Importance}{TKD~3.3.1}
  \SuffixGuideDefinition{'a'}{Augmentative}{Indicates that the noun
    is bigger, more important, or more powerful than it would be
    without the suffix.}{}
  \SuffixGuideDefinition{Hom}{Diminutive}{Indicates that the noun is
    smaller, less important, or less powerful than it would be
    without the suffix.}{}
  \SuffixGuideDefinition[TKDa]{oy}{Endearment}{A \B{'} is probably
    inserted before this suffix, if the noun it attaches to ends in a
    vowel.}{}
%
%
%\hline
\SuffixGuideType{2}{Number}{TKD~3.3.2}
  \SuffixGuideDefinition{pu'}{Plural for beings capable of using language}{}{}
  \SuffixGuideDefinition{Du'}{Plural for body parts}{}{}
  \SuffixGuideDefinition{mey}{Plural, general usage}{The suffix
    \B{-mey} carries a notion of ``scattered all about'' when applied
    to words that normally take the suffix \B{-pu'} (\B{puq\-mey}
    \I{children all over the place} versus \B{puq\-pu'}
    \I{children}). The same thing happens when the \B{-mey} plural
    suffix is applied to the singular form of a noun that is
    irregularly pluralized (\B{DoS\-mey} \I{targets scattered all about}
    versus \B{ray'} \I{targets}).}{}
%
%
%\hline
\SuffixGuideType{3}{Qualification}{TKD~3.3.3}
  \SuffixGuideDefinition{qoq}{So-called}{}{}
  \SuffixGuideDefinition{Hey}{Apparent}{}{}
  \SuffixGuideDefinition{na'}{Definite}{}{}
%
%
%\hline
\SuffixGuideType{4}{Possession/Specification}{TKD~3.3.4}

%\SuffixGuideSpacer
  \SuffixGuideSubtype{Possessives for Beings Capable of Using Language}
  \SuffixGuideDefinition{wI'}{My}{}{}
  \SuffixGuideDefinition{ma'}{Our}{}{}
  \SuffixGuideDefinition{lI'}{Your}{}{}
  \SuffixGuideDefinition{ra'}{Your (plural)}{These suffixes are used
    to indicate possessives when referring to beings capable of
    speech, e.g.\ \B{jup\-wI'} \I{my friend;} \B{be'\-nal\-lI'} \I{your
    wife;} \B{puq\-ma'} \I{our child.} The general possessive suffixes
    may also be used, but they are considered derogatory; \B{juH\-wIj}
    for \I{my lord} borders on the taboo.}{}

%\SuffixGuideSpacer
  \SuffixGuideSubtype{Possessives, General Usage}
  \SuffixGuideDefinition{wIj}{My}{}{}
  \SuffixGuideDefinition{maj}{Our}{}{}
  \SuffixGuideDefinition{lIj}{Your}{}{}
  \SuffixGuideDefinition{raj}{Your (plural)}{}{}
  \SuffixGuideDefinition{Daj}{His, her, its}{}{}
  \SuffixGuideDefinition{chaj}{Their}{Though grammatically correct, it
    is considered derogatory to use these general possessive suffixes
    when referring to first or second persons that are are capable of
    language (i.e.\ \I{I, we, you} or \I{you [plural]}).}{}

%\SuffixGuideSpacer
  \SuffixGuideSubtype{Specification}
  \SuffixGuideDefinition{vam}{This}{Indicates that the noun refers to
    an object which is nearby or which is the topic of the
    conversation, e.g.\ \B{nuH\-vam} \I{this weapon (near me as I
    speak);} \B{yuQ\-vam} \I{this planet (that we have been talking
    about).}}{}
  \SuffixGuideDefinition{vetlh}{That}{Indicates that the noun refers
    to an object which is not nearby or which is being brought up
    again as topic of the conversation, e.g.\ \B{nuH\-vetlh} \I{that
    weapon (over there);} \B{yuQ\-vetlh} \I{that planet (as opposed to
    the one we were just talking about).}}{}
%
%
%\hline
\SuffixGuideType{5}{Syntactic Markers}{TKD~3.3.5}
  \SuffixGuideDefinition{Daq}{Locative}{Corresponds to English
    prepositions such as: \I{at, to, in, on,} depending on
    context, e.g.\ \B{juH\-Daq jIH} \I{I'm at home;} \B{meH\-Daq yI\-qet}
    \I{Run to the bridge!} It is not used in the abstract sense of
    \I{in English}. Some nouns never take this suffix (notably \B{na\-Dev}
    \I{hereabouts,} \B{pa'} \I{thereabouts,} and \B{Dat} \I{everywhere}).
    \B{pa'} can mean both \I{room} and \I{thereabouts} but is distinct
    in use, e.g.\ \B{pa' yI\-jaH} \I{Go over there!} \B{pa'\-Daq yI\-jaH}
    \I{Go to the room!} Some verbs include locative notions, and take
    a place as object rather than locative, e.g.\ \B{Duj ghoS\-taH}
    \I{It is approaching the ship.}}{}
  \SuffixGuideDefinition{vo'}{From}{Similar to \B{-Daq} but used only
    for actions directed away from a place, e.g. \B{pa'\-vo' yI\-jaH}
    \I{Leave the room!} This suffix can not be used in the abstract
    sense of \I{from the past.}}{}
  \SuffixGuideDefinition{mo'}{Due to, because of}{E.g.\ \B{HIq\-mo'
    bI'\-uH} \I{You have a hangover because of the alcohol.} Note that
    there is also a verb suffix \B{-mo'} with the same meaning.}{}
  \SuffixGuideDefinition{vaD}{For, intended for}{Indicates the
    indirect object, or beneficiary of the action. The indirect object
    precedes the object, e.g.\ \B{chaH\-vaD Soj qem yaS} \I{The officer
    brings them food;} \B{Qu'\-vaD lI' De'\-vam} \I{This information is
    useful for the mission.}}{TKDa~6.8}
  \SuffixGuideDefinition{'e'}{Topic}{Emphasizes that the noun it is
    attached to is the topic of the sentence. Is also used to mark the
    head noun of a relative clause (see verb suffix \B{-bogh} \I{which}
    above). If the object of a sentence is marked with \B{-'e'} adverbials
    may come after the object instead at the beginning of the
    sentence.}{TKDa~6.7}

%[eof]
