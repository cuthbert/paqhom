% Sammanfattning:
%
% This is the preface text. It is included by the "main.tex", and
% cannot my be compiled by itself (since it lacks the neccessary
% "\begin{document}...\end{document} for this).
%
%%%%%%%%%%%%%%%%%%%%%%%%%%%%%%%%%%%%%%
%% Add stress to the pronunciation. %%
%%%%%%%%%%%%%%%%%%%%%%%%%%%%%%%%%%%%%%
%

% *** Preface

\parindent=1em                                  %% en fyrkants indrag
\lhead[\large\B{Källförkortningar}]{}%          %% Left header col
\rhead[]{\large\B{Inledning \& Ordklassförkortningar}}% Right head col

\raggedbottom
\header{Inledning}

\noindent Den här ordboken består av en sammanställning av alla de
klingonska ord som återfinns i upphovsmannens (Dr.\ Marc Okrand) olika
verk. Eftersom intentionen med den här boken varit att skapa ett
praktiskt referensverk i fickformat, snarare än ett fullständig
beskrivning av språket, antas det att du som läsare redan har
grundläggande kunskaper om språkets grammatik och struktur. Om så inte
är fallet så vill vi uppmana dig att läsa Okrands huvudsakliga verk i
ämnet, \I{The Klingon Dictionary} (Pocket Books, New York, 1992,
ISBN~0-671-74559-X).

Bokens innehåll har genererats ifrån en databas innehållande
\fromkliwords{} klingonska ord, och \tokliwords{} svenska
uppslagsord. Databasen skapades ursprungligen under hösten 1997 och
har sedan dess kontinuenligt uppdaterats. Om du skulle finna fel eller
upptäcka att väsentlig information saknas någonstans, eller om du vill
veta mer om \I{Kling\-on\-ska Aka\-demi\-en} och dess projekt,
kontakta gärna oss på följande adress:

%\begin{multicols}{2}\begin{center}

\vspace{3mm}%
\noindent\parbox[t]{.5\textwidth}{%
  \rule{1.25cm}{0mm}\I{Klingonska Akademien}\\
  \rule{1.25cm}{0mm}Villavägen 33, 2tr\\
  \rule{1.25cm}{0mm}S-752~36~Uppsala, Sweden%
}%
\parbox[t]{.5\textwidth}{%
  \rule{1.75cm}{0mm}\I{www.klingonska.org}\\
  \rule{1.75cm}{0mm}\I{zrajm@klingonska.org}\\
  \rule{1.75cm}{0mm}+46~(0)18~500~911%
}
%\end{center}\end{multicols}


\header{Ordklassförkortningar}

\noindent Den här boken använder samma ordklassförkortningar som TKD,
med ett tillägg --~ordklassen ''namn'', vilken används för indikera
personnamn. Endast namn som återfinns i okrandiska källor har
inkluderats i ordboken.

\begin{center}\begin{tabular}{lll}
\I{v}    & verb          & {\small [TKD~4]}   \\
\I{n}    & substantiv    & {\small [TKD~3]}   \\
\I{name} & namn          & {\small [TKD~5.6]} \\
\I{pro}  & pronomen      & {\small [TKD~5.1]} \\
\I{adv}  & adverb        & {\small [TKD~5.4]} \\
\I{num}  & nummer        & {\small [TKD~5.2]} \\
\I{excl} & interjektion  & {\small [TKD~5.5]} \\
\I{ques} & frågeord      & {\small [TKD~6.4]} \\
\I{conj} & konjuktion    & {\small [TKD~5.3]}
\end{tabular}\end{center}


%s:v      |verb        
%s:s      |substantiv  
%s:namn   |namn        
%s:pro    |pronomen    
%s:adv    |adverb      
%s:räkn   |räkneord    
%s:interj |interjektion
%s:fråg   |frågeord    
%s:konj   |konjunktion 



\header{Källförkortningar}

\noindent Alla ord på klingonska i den här ordboken har bekräftats i
kanoniska källor, och listan nedan är en förteckning över de vanligast
förekommande källorna. Ytterligare andra källor förekommer, men endast
så sällan att det inte varit värt att ange dem här, i förekommande
fall är istället källreferensen fullt utskriven. Undantaget är dock MOs
postningar till Usenet, där källan endast anges som ''News'' och det
datumet då postningen gjordes i \mbox{\I{ÅÅÅÅ-MM-DD}}-format. Om mot
förmodan finner en källa du inte kan placera så rekommenderas artikeln
''Archive of Okrandian Canon'' som finns att tillgå på våran webbplats
(\I{www.klingonska.org}) ytterligare information kan också hittas i
\I{The Klingon Mailing List FAQ} (\I{www.bigfoot.com/\linebreak
\unspace\lemma{}\,dspeers/klingon/faq.htm}).



\vspace{7mm}
\begin{center}
\begin{tabular}{lll}
\B{BoP} & \I{Klingon Bird of Prey Cutaway Poster}\rule[-3mm]{0mm}{3mm}\\
\B{CK}  & \I{Conversational Klingon}  (ljudinspelning)\rule[-3mm]{0mm}{3mm}\\
\B{HQ}  & \I{\B{HolQeD}}              (\I{Klingon Language Institute}s tidning)\rule[-3mm]{0mm}{3mm}\\
\B{KCD} & \I{Star Trek: Klingon!}       (datorspel \& språklabb)\rule[-3mm]{0mm}{3mm}\\
\B{KGT} & \I{Klingon for the Galactic Traveler} (bok)\rule[-3mm]{0mm}{3mm}\\
\B{KLI} & \I{The Klingon Language Institute}\rule[-3mm]{0mm}{3mm}\\
\B{MO}  & Marc Okrand                 (språkets skapare)\rule[-3mm]{0mm}{3mm}\\
\B{PK}  & \I{Power Klingon}           (ljudinspelning)\rule[-3mm]{0mm}{3mm}\\
\B{S\#} & \I{SkyBox Trading Card S\#}\rule[-3mm]{0mm}{3mm}\\
\B{ST5} & \I{Star Trek V: The Final Frontier} (film)\rule[-3mm]{0mm}{3mm}\\
\B{STE} & \I{Star Trek Encyclopedia} (bok)\rule[-3mm]{0mm}{3mm}\\
 & (använd endast för stavning av engelska ord)\rule[-3mm]{0mm}{3mm}\\
\B{TKD} & \I{The Klingon Dictionary} (bok)\rule[-3mm]{0mm}{3mm}\\
\B{TKDa}& Addendum to \I{The Klingon Dictionary}\rule[-3mm]{0mm}{3mm}\\
\B{TKW} & \I{The Klingon Way} (bok)\rule[-3mm]{0mm}{3mm}\\
\B{E-K} & förekommer endast i källans engelsk-klingonska del\rule[-3mm]{0mm}{3mm}\\
\B{K-E} & förekommer endast i källans klingonsk-engelska del\rule[-3mm]{0mm}{3mm}\\
\end{tabular}
\end{center}




\newpage

\lhead[\large\B{Alfabet och uttal \& Betoning}]{}% Left header col
\rhead[]{\large\B{Alfabet och uttal}}% Right header col
%\chead[\large\B{Alphabet and Pronunciation}]%
%  {\large\B{Alphabet and Pronunciation}} % Remove page header

\header{Alfabet och uttal}

\noindent Ordningen på det klingonska alfabetet är som följer:

\begin{center}
\B{a}, \B{b}, \B{ch}, \B{D}, \B{e}, \B{gh}, \B{H}, \B{I}, \B{j},
\B{l}, \B{m}, \B{n}, \B{ng}, \\
\B{o}, \B{p}, \B{q}, \B{Q}, \B{r}, \B{S}, \B{t}, \B{tlh}, \B{u},
\B{v}, \B{w}, \B{y}, \B{'}
\end{center}

\noindent Iakta särskilt at \B{ch}, \B{gh}, \B{ng}, \B{tlh} och \B{'}
anses vara bokstäver i sin egen rätt, ett resultat av detta är att
ordet \B{nob} kommer \I{före} ordet \B{ngab} i en lista sorterad i
klingonsk bokstavsordning. Bokstäverna \B{q} och \B{Q} representerar
två olika ljud, och sorteras också som två separata bokstäver.

Det som här följer är endast en grov guide till klingonsskans uttal,
för en mer detaljerad beskrivning se TKD sektion 1.

\vspace{3mm}

\phonexpl{a}%
  {\textipa{[A]} As in \I{ps\U{a}lm} or \I{p\U{a}}, never as in
  \I{cr\U{a}b\U{a}pple}.}{open back unrounded}


\phonexpl{b}%
  {\textipa{[b]} As in \I{\U{b}ronchitis}, \I{gaze\U{b}o} or
  \I{\U{b}ribe}.}{voiced bilabial stop}

\phonexpl{ch}%
  {\textipa{[\t{tS}]} As in \I{\U{ch}ew} or
  \I{arti\U{ch}oke}.}{voiceless postalveolar affricate}

\phonexpl{D}%
  {\textipa{[\:d]} As in Swedish \I{vä\U{rd}} (host), further back
  than English \I{d} as in \I{\U{d}ream} or \I{an\U{d}roid}. Let the
  tongue touch halfway between the teeth and the soft palate.}{voiced
  retroflex stop}

\phonexpl{e}%
  {\textipa{[E]} As in \I{s\U{e}nsor} or \I{p\U{e}t}.}{open-mid front
  unrounded}

\phonexpl{gh}%
  {\textipa{[G]} Put tongue as if to say \I{gobble}, but relax and
  hum. Almost the same as \B{H} but voiced.}{voiced velar fricative}

\phonexpl{H}%
  {\textipa{[x]} As in the name of the german composer
  \I{Ba\U{ch}}. Very strong and coarse. Similar to \B{gh} but without
  humming.}{voiceless velar fricative}

\phonexpl{I}%
  {\textipa{[I]} As in \I{m\U{i}sf\U{i}t} or \I{p\U{i}t}.}{semi-close
  front unrounded}


\phonexpl{j}%
  {\textipa{[\t{dZ}]} As in \I{\U{j}unk} (with an initial
  \I{d}-sound), never as in French \I{jour}.}{voiced postalveolar
  affricate}

\phonexpl{l}%
  {\textipa{[l]} As in \I{\U{l}unge} or \I{a\U{l}chemy}.}{voiced
  alveolar lateral approximant}

\phonexpl{m}%
  {\textipa{[m]} As in \I{\U{m}ud} or \I{pneu\U{m}atic}.}{voiced
  bilabial nasal}

\phonexpl{n}%
  {\textipa{[n]} As in \I{\U{n}ectarine} or \I{su\U{n}spot}.}{voiced
  alveolar nasal}

\phonexpl{ng}%
  {\textipa{[N]} As in \I{furlo\U{ng}} or \I{thi\U{ng}}, never as in
  \I{e\U{ng}ulf}. Also occurs at the beginning of syllables.}{voiced
  velar nasal}

\phonexpl{o}%
  {\textipa{[o]} As in \I{g\U{o}} or \I{m\U{o}saic}.}{close-mid back
  rounded}

\phonexpl{p}%
  {\textipa{[p\super h]} As in \I{\U{p}arallax} or \I{o\U{pp}obrium},
  always with a strong puff or pop, never laxly.}{aspirated voiceless
  bilabial stop}

\phonexpl{q}%
  {\textipa{[q\super h]} Similar to \I{k} in \I{kumquat}, but further
  back. The tongue should touch the uvula while saying this. A puff of
  air should accompany the sound.}{aspirated voiceless uvular stop}

\phonexpl{Q}%
  {\textipa{[\t{qX}]} A harder variant of \B{q}, very strong and
  raspy.}{voiceless uvular affricate}

\phonexpl{r}%
  {\textipa{[r]} or \textipa{[\*r]} A trilled \I{r} using the tip of
  the tongue, as in Swedish \I{\U{r}ö\U{r}} (pipe, tube) if properly
  articulated.}{voiced alveolar trill}

%\phonexpl{S}%
%  {\textipa{[\:s]} As in Swedish \I{fo\U{rs}} (rapid stream) or as an
%  English \I{s} articulated with the tongue in the Klingon \B{D}
%  position.}{voiceless retroflex fricative}

\phonexpl{S}%
  {\textipa{[\:s]} As in Swedish \I{mothå\U{rs}} (against the
  predominant direction of hair growth e.g. on a pet) or as an English
  \I{s} articulated with the tongue in the Klingon \B{D}
  position.}{voiceless retroflex fricative}

\phonexpl{t}%
  {\textipa{[t\super h]} As in \I{\U{t}arpaulin} or
  \I{cri\U{t}ique}. It is accompanied by a puff of air.}{aspirated
  voiceless dental stop}

\phonexpl{tlh}%
  {\textipa{[\t{t\textbeltl}]} To learn how to say this Klingon sound,
  first say \I{l}, then keep your tongue in the same position and
  exhale. Now repeat this, but let the air build up pressure behind
  your tongue before releasing it. The resulting sound should be
  voiceless, and you should be able to feel the air escape quite
  forcefully on either side of your tongue.}{voiceless alveolar
  lateral affricate}

\phonexpl{u}%
  {\textipa{[u]} As in \I{gn\U{u}}, \I{pr\U{u}ne} or \I{s\U{oo}n},
  never as in \I{b\U{u}t} or \I{c\U{u}te}.}{close back rounded}

\phonexpl{v}%
  {\textipa{[v]} As in \I{\U{v}ulgar} or
  \I{demonstrati\U{v}e}.}{voiced labiodental fricative}

\phonexpl{w}%
  {\textipa{[w]} As in \I{\U{w}orry\U{w}art} or \I{co\U{w}}.}{voiced
  rounded labiovelar approximant}

\phonexpl{y}%
  {\textipa{[j]} As in \I{\U{y}odel} or \I{jo\U{y}}.}{voiced palatal
  central approximant}

\phonexpl{'}%
  {\textipa{[P]} As in the abrupt cut-off of sound in \I{uh-oh} or
  \mbox{\I{unh-unh}} meaning ``no''. At the end of a word this sound
  is usually followed by a soft echo of the preceeding sound.}{glottal
  stop}


\header{Betoning}

\noindent Verb betonas alltid på den sista stavelsen i huvudordet,
emedan det första suffixet är alltid obetonat. Om det därefter kommer
ett eller flera suffix som slutar på \B{'} så betonas de också. Det
finns dock ett undantag; Om en talare vill ge eftertryck åt ett
särskilt suffix (oftast ett frågesuffix eller ett suffix som används
för att negera eller poängtera något) så kan betoningen istället
flyttas till dessa suffix och lämna övriga stavelser obetonade. Verb
som används adjektivistiskt betonas på samma sätt som andra verb.

Substantiv betonas vanligtvis på stammens sista stavelse. Skulle det
dock råka en eller fler stavelser i ordet med suffix, som slutar på
\B{'} så betonas den stavelsen istället, om det det finns fler än en sådan
stavelse så betonas de allihop. Substantiv som konstruerats av ett
verb och suffixet \B{-wI'} eller \B{-ghach}, betonas som om de vore
substantiv.

\clearpage
