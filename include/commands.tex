\usepackage[T1]{fontenc}                     % allow åäö
\usepackage{vmargin}                         % to set margins
\usepackage{fancyhdr}                        % load fancy headers
\usepackage{fixmarks}                        % header patch for 2 columns
\usepackage{multicol}                        % two column package
\usepackage{rotate}                          % to rotate reference tables 
\usepackage{graphics}                        % used by the \lemma{} command
\usepackage{amssymb}                         % black triangles (cover)
\usepackage{colortbl}                        % color tables (cover)
\usepackage{hhline}                          % horiz lines for colortable (cover)
\usepackage{multirow}                        % for tables (cover)
\usepackage{zrajm}                           % my own stuff

\usepackage{ifthen}                          % enable \ifthenelse & \whiledo
\usepackage{calc}                            % simple arithmetics 
\usepackage{tipa}                            % i.p.a. typeface
\usepackage{avant}                           % typeface "Avantgarde" (sans)
%\newcommand{\sans}{\usefont{OT1}{pag}{m}{n}}% åäö doesn't show using this
\newcommand{\sans}{\usefont{T1}{pag}{m}{n}}  % åäö works ok
\usepackage{charter}                         % typeface "Charter" (serif)
% Last selected typeface becomes the document default.


\renewcommand{\small}{\fontsize{12pt}{14.4pt}\selectfont}     % small text
\renewcommand{\normalsize}{\fontsize{16pt}{19.2pt}\selectfont}% normal text
\renewcommand{\large}{\fontsize{20pt}{24pt}\selectfont}   % headers/footers
\renewcommand{\Large}{\fontsize{20pt}{24pt}\selectfont}   % sectional headings
\renewcommand{\LARGE}{\fontsize{24pt}{28.8pt}\selectfont} % not used
\renewcommand{\huge}{\fontsize{32pt}{38.4pt}\selectfont}  % prefix table header
\renewcommand{\Huge}{\fontsize{32pt}{38.4pt}\selectfont}  % front cover name




% ================================
% English-Klingon Dictionary Entry
% ================================
%
%   #1 = English lookup word
%   #2 = Klingon word
%   #3 = Part-of-speech
%   #4 = 'slang' or 'regional' etc.
%   #5 = English explanation of the word
%        (if Ø then the explanation equals the lookup word)
%        (the lookup word is replaced by \lemma{}, which generates a tilde)
%
\newlength{\argwidth}
\newcommand{\lueng}[5]{%
  \par%
  \setlength{\hangindent}{1em / 2}%        % set hanging indent
  {\sans\B{#1}}%                           % english lookup word (sansserif)
  \markboth{#1}{#1}%                       % page header info
  \ \textit{#3}%                           % part-of-speech
  \ifthenelse{\equal{#4}{}}{}%             % regional/slang info (if any)
    {{\small\ #4}}%                        % --||--
  \ \B{#2}%                                % klingon word
  \settowidth{\argwidth}{#5}%              % english definition
  \ifthenelse{\lengthtest{\argwidth>0cm}}% % --||--
    {\ #5}{}%                              % --||--
}

% ================================
% Klingon-English Dictionary Entry
% ================================
%
%   #1 = Klingon (lookup) word
%   #2 = part-of-speech
%   #3 = 'slang' or 'regional' etc.
%   #4 = English explanation of the word
%   #5 = source
%        (if Ø then source=TKD)
%
\newcommand{\lukli}[5]{%
  \par%
  \setlength{\hangindent}{1em / 2}%        % set hanging indent
  {\sans\B{#1}}%                           % klingon lookup word (sansserif)
  \markboth{#1}{#1}%                       % page header info
  \ \textit{#2}%                           % part of speech
  \ifthenelse{\equal{#3}{}}%               % regional/slang info (if any)
    {}{{\small\ #3}}%                      % --||--
  \ #4%                                    % english definition
  \ifthenelse{\equal{#5}{}}%               % source (if not TKD)
    {}{{\small\ [#5]}}%                    % --||--
}


% used when going from one letter to
% another in the dictionary
% [2001-03-03] Zrajm C Akfohg
\newcommand{\newletter}{%
  \par%
%  \vspace{\baselineskip}%
  \vspace{\baselineskip}%
}




% section headings (how could the `\section'
% command could be made to do this instead?)
\newcommand{\header}[1]{%
  \vspace{7mm}%
  {\noindent\Large\B{#1}}%
  \vspace{3mm}%
}

% used for outputting the description of
% the klingon sounds in the preface
\newcommand{\phonexpl}[3]{%
  \hang{%
    \sans\B{#1}%
  } #2% {\small [#3]}%    % phonetic descriptions turned off
}


% =================================================================
% Set Values for \NumElementList, \NounSuffixList & \VerbSuffixList
% =================================================================
% [2000-10-25] Written by Zrajm C Akfohg.
%
% This routine sets the values associated with tables to nice
% defaults - also my own \sufcolwidth length is set, which is
% needed by all the above mentioned affix listing routines.
%
\newlength{\sufcolwidth}%
\newcommand{\PrepAffixLists}{%
  \fboxsep=0pt%
  \tabcolsep=0mm%
  \doublerulesep=0mm%
  \arrayrulewidth=.6mm%
  \parindent=0mm%
    \setlength{\sufcolwidth}{\textheight}%
    \addtolength{\sufcolwidth}{-11\arrayrulewidth}%
    \setlength{\sufcolwidth}{.1\sufcolwidth}%
}



% ===========================================================
% Title strings for the numeric element list and affix lists.
% ===========================================================
% [2001-12-12] Written by Zrajm C Akfohg.
%
% This tests if my \dictlanguage{} "variable" equals "swedish",
% if it does, the titles of the affix tables are set to their
% Swedish translations, otherwise the English titles are used.
%
\ifthenelse{\equal{\dictlanguage}{swedish}}{%
    %%%%%%%%%%%%%%%%%%
    % for the suffix tables
    \newcommand{\VerbSuffixListTitle}{Verbsuffix (36 st)}%
    \newcommand{\NounSuffixListTitle}{Substantivsuffix (26 st)}%
    \newcommand{\NumElementListTitle}{Numeriska element}%
    %%%%%%%%%%%%%%%%%%
    % for the prefix table
    \newcommand{\VerbPrefixTableTitleSubject}{%
      \centering\Huge%
      \parbox[c]{1em}{%
        \centering\B{S\\U\\B\\J\\E\\K\\T}%
      }%
    }%
    \newcommand{\VerbPrefixTableTitleObject}{
      \Huge\B{\hspace{\stretch{3}}%
        O\hspace{\stretch{1}}%
        B\hspace{\stretch{1}}%
        J\hspace{\stretch{1}}%
        E\hspace{\stretch{1}}%
        K\hspace{\stretch{1}}%
        T\hspace{\stretch{3}}%
        \raisebox{\depthof{g} * \real{-2}}{%
          \rule{0mm}{(\depthof{g} * 4) + \heightof{O}}%
        }%
      }%
    }%
}{%
    %%%%%%%%%%%%%%%%%%
    % for the suffix tables
    \newcommand{\VerbSuffixListTitle}{Verb Suffixes (36 total)}%
    \newcommand{\NounSuffixListTitle}{Noun Suffixes (26 total)}%
    \newcommand{\NumElementListTitle}{Numeric Elements}%
    %%%%%%%%%%%%%%%%%%
    % for the prefix table
    \newcommand{\VerbPrefixTableTitleSubject}{%
      \centering\Huge%
      \parbox[c]{1em}{%
        \centering\B{S\\U\\B\\J\\E\\C\\T}%
      }%
    }%
    \newcommand{\VerbPrefixTableTitleObject}{
      \Huge\B{\hspace{\stretch{3}}%
        O\hspace{\stretch{1}}%
        B\hspace{\stretch{1}}%
        J\hspace{\stretch{1}}%
        E\hspace{\stretch{1}}%
        C\hspace{\stretch{1}}%
        T\hspace{\stretch{3}}%
        \raisebox{\depthof{g} * \real{-2}}{%
          \rule{0mm}{(\depthof{g} * 4) + \heightof{O}}%
        }%
      }%
    }%
}


% ======================
% Numerical Element List
% ======================
% [2000-10-25] Written by Zrajm C Akfohg.
%
% This routine produces a reference list of the Klingon
% number-forming elements (without English translations).
%
\newcommand{\NumElementList}{%
\normalsize\begin{tabular}[b]{cccccl}
  &\multicolumn{5}{l}{{\B{\Large\NumElementListTitle}}\vspace{1.5mm}}%
\\\cline{1-5}%
%  \vline&\parbox[c][2em]{\sufcolwidth}{~}&%
%  \vline&\parbox[c][2em]{\sufcolwidth}{~}&\vline
%  &\hfill
%\\\cline{1-5}%
%  \vline&\B{-maH}&
%  \vline&\B{-DIch}&\vline
  \vline&\parbox[c]{\sufcolwidth}{\centering\B{-maH}}&
  \vline&\parbox[c]{\sufcolwidth}{\centering\B{-DIch}}&\vline
\\%
  \vline&\B{-vatlh}&
  \vline&\B{-logh}&\vline
\\\cline{3-5}%
  \vline&\B{-SaD}&\vline
\\%
  \vline&\B{-SanID}&\vline
\\%
  \vline&\B{-netlh}&\vline
\\%
  \vline&\B{-bIp}&\vline
\\%
  \vline&\B{-'uy'}&\vline
\\\cline{1-3}%
\end{tabular}}




% ================
% Noun Suffix List
% ================
% [2000-10-25] Written by Zrajm C Akfohg.
%
% This routine produces a reference list of the Klingon
% noun suffixes and their type (without English translations).
%
\newcommand{\NounSuffixList}{%
\normalsize\begin{tabular}[b]{ccccccccccc}
  &\multicolumn{10}{l}{{\B{\Large\NounSuffixListTitle}}\vspace{1.5mm}}%
\\\hline%
  \vline&\parbox[c][1.5em]{\sufcolwidth}{\centering\Large\B{1}}&%
  \vline&\parbox[c][1.5em]{\sufcolwidth}{\centering\Large\B{2}}&%
  \vline&\parbox[c][1.5em]{\sufcolwidth}{\centering\Large\B{3}}&%
  \vline&\parbox[c][1.5em]{\sufcolwidth}{\centering\Large\B{4}}&%
  \vline&\parbox[c][1.5em]{\sufcolwidth}{\centering\Large\B{5}}&\vline%
\\\hline%
  \vline&\B{-'a'}&%
  \vline&\B{-pu'}&%
  \vline&\B{-qoq}&%
  \vline&\B{-wIj}&%
  \vline&\B{-Daq}&\vline%
\\%
  \vline&\B{-Hom}&%
  \vline&\B{-Du'}&%
  \vline&\B{-Hey}&%
  \vline&\B{-wI'}&%
  \vline&\B{-vo'}&\vline%
\\%
  \vline&\B{-oy}&%
  \vline&\B{-mey}&%
  \vline&\B{-na'}&%
  \vline&\B{-maj}&%
  \vline&\B{-mo'}&\vline%
\\\cline{1-7}%
  &&&&&&%
  \vline&\B{-ma'}&%
  \vline&\B{-vaD}&\vline%
\\%
  &&&&&&%
  \vline&\B{-lIj}&%
  \vline&\B{-'e'}&\vline%
\\\cline{9-11}%
  &&&&&&%
  \vline&\B{-lI'}&%
  \vline&&%
\\%
  &&&&&&%
  \vline&\B{-raj}&%
  \vline&&%
\\%
  &&&&&&%
  \vline&\B{-ra'}&%
  \vline&&%
\\%
  &&&&&&%
  \vline&\B{-Daj}&%
  \vline&&%
\\%
  &&&&&&%
  \vline&\B{-chaj}&%
  \vline&&%
\\%
  &&&&&&%
  \vline&\B{-vam}&%
  \vline&&%
\\%
  &&&&&&%
  \vline&\B{-vetlh}&%
  \vline&&%
\\\cline{7-9}%
\end{tabular}}



% ================
% Verb Suffix List
% ================
% [2000-10-25] Written by Zrajm C Akfohg.
%
% This routine produces a reference list of the Klingon
% verb suffixes and their type (without English translations).
%
\newcommand{\VerbSuffixList}{%
\normalsize\begin{tabular}[b]{ccccccccccccccccccccc}
  &\multicolumn{20}{l}{{\B{\Large\VerbSuffixListTitle}}\vspace{1.5mm}}%
\\\cline{1-21}%
  \vline&\parbox[c][1.5em]{\sufcolwidth}{\centering\Large\B{1}}&%
  \vline&\parbox[c][1.5em]{\sufcolwidth}{\centering\Large\B{2}}&%
  \vline&\parbox[c][1.5em]{\sufcolwidth}{\centering\Large\B{3}}&%
  \vline&\parbox[c][1.5em]{\sufcolwidth}{\centering\Large\B{4}}&%
  \vline&\parbox[c][1.5em]{\sufcolwidth}{\centering\Large\B{5}}&%
  \vline&\parbox[c][1.5em]{\sufcolwidth}{\centering\Large\B{6}}&%
  \vline&\parbox[c][1.5em]{\sufcolwidth}{\centering\Large\B{7}}&%
  \vline&\parbox[c][1.5em]{\sufcolwidth}{\centering\Large\B{8}}&%
  \vline&\parbox[c][1.5em]{\sufcolwidth}{\centering\Large\B{9}}&%
  \vline&\parbox[c][1.5em]{\sufcolwidth}{\centering\Large\B{R}}&\vline%
\\\hline%
  \vline&\B{-'egh}&%
  \vline&\B{-nIS}&%
  \vline&\B{-choH}&%
  \vline&\B{-moH}&%
  \vline&\B{-lu'}&%
  \vline&\B{-chu'}&%
  \vline&\B{-pu'}&%
  \vline&\B{-neS}&%
  \vline&\B{-DI'}&%
  \vline&\B{-be'}&\vline%
\\\cline{7-9}\cline{15-17}%
  \vline&\B{-chuq}&%
  \vline&\B{-qang}&%
  \vline&\B{-qa'}&%
  \vline&&%
  \vline&\B{-laH}&%
  \vline&\B{-bej}&%
  \vline&\B{-ta'}&%
  \vline&&%
  \vline&\B{-chugh}&%
  \vline&\B{-Qo'}&\vline%
\\\cline{1-3}\cline{5-7}\cline{9-11}&&%
  \vline&\B{-rup}&%
  \vline&&&&&&%
  \vline&\B{-law'}&%
  \vline&\B{-taH}&%
  \vline&&%
  \vline&\B{-pa'}&%
  \vline&\B{-Ha'}&\vline%
\\&&%
  \vline&\B{-beH}&%
  \vline&&&&&&%
  \vline&\B{-ba'}&%
  \vline&\B{-lI'}&%
  \vline&&%
  \vline&\B{-vIS}&%
  \vline&\B{-qu'}&\vline%
\\\cline{11-15}\cline{19-21}&&%
  \vline&\B{-vIp}&%
  \vline&&&&&&&&&&&&%
  \vline&\B{-bogh}&%
  \vline&&%
\\\cline{3-5}&&&&&&&&&&&&&&&&%
  \vline&\B{-meH}&%
  \vline&&%
\\&&&&&&&&&&&&&&&&%
  \vline&\B{-mo'}&%
  \vline&&%
\\&&&&&&&&&&&&&&&&%
  \vline&\B{-'a'}&%
  \vline&&%
\\&&&&&&&&&&&&&&&&%
  \vline&\B{-jaj}&%
  \vline&&%
\\&&&&&&&&&&&&&&&&%
  \vline&\B{-wI'}&%
  \vline&&%
\\&&&&&&&&&&&&&&&&%
  \vline&\B{-ghach}&%
  \vline&&%
\\\cline{17-19}%
\end{tabular}}


% ================
% Verb Prefix List
% ================
% [2001-10-25] Written by Zrajm C Akfohg.
%
% This routine produces a reference table of the Klingon verb
% prefixes (without English translations).
%
%%%%%%%%%%%%%%%%%%
%% needed by the prefix table
\newcommand{\VerbPrefixTablePrefixCell}[1]{\B{#1}&}
\newcommand{\VerbPrefixTableEmptyCell}[1]{%
  \multicolumn{1}{#1>{\columncolor[gray]{.8}}c|}{}&%
}
\newcommand{\VerbPrefixTablePronounCell}[1]{\B{#1}&}
%%%%%%%%%%%%%%%%%%%%%%%%%%%%
\newcommand{\VerbPrefixTable}{%
%
\Large%
\parindent=0mm%
%
\setlength{\fboxsep}{0cm}%           %% padding inside fbox
\setlength{\fboxrule}{.6mm}%         %% line thickness of boxes
\setlength{\tabcolsep}{2mm}%         %% horizontal padding in cells
\setlength{\arrayrulewidth}{.6mm}%   %% line thickness in table
%\renewcommand{\arraystretch}{1.3}%   %% height of table rows (broad back)
\renewcommand{\arraystretch}{1.1}%   %% height of table rows (narrow back)
\setlength{\extrarowheight}{1mm}%    %% add extra space above text in table
%
\fbox{\begin{tabular}{|c|c|c|c|c|c|c|c|r|}%
\hline%
  \multicolumn{7}{|c|}{\VerbPrefixTableTitleObject}&
  \multicolumn{2}{c|}{}%
\\\hhline{*{7}{|-}|~~|}%
  \VerbPrefixTablePrefixCell{jIH}%
  \VerbPrefixTablePrefixCell{maH}%
  \VerbPrefixTablePrefixCell{SoH}%
  \VerbPrefixTablePrefixCell{tlhIH}%
  \VerbPrefixTablePrefixCell{%
    \parbox[c][\height][c]{\widthof{ghaH}}{\centering
      \vspace{2mm}%
      ghaH\\'oH%
      \vspace{2mm}%
    }%
  }
  \VerbPrefixTablePrefixCell{%
    \parbox{\widthof{chaH}}{%
      \centering chaH\\bIH%
    }%
  }
  \VerbPrefixTablePrefixCell{pagh}%
  \multicolumn{2}{c|}{}%
\\\hhline{*{9}{|-}|}
  \VerbPrefixTablePrefixCell{HI-}%
  \VerbPrefixTablePrefixCell{gho-}%
  \VerbPrefixTableEmptyCell{}%
  \VerbPrefixTableEmptyCell{}%
  \VerbPrefixTablePrefixCell{yI-}%
  \VerbPrefixTablePrefixCell{tI-}%
  \VerbPrefixTablePrefixCell{yI-}%
  \VerbPrefixTablePronounCell{SoH (ra')}%
  \multirow{9}{2em}{%
    \VerbPrefixTableTitleSubject%
  }%
\\\hhline{*{8}{|-}|~|}
  \VerbPrefixTablePrefixCell{HI-}%
  \VerbPrefixTablePrefixCell{gho-}%
  \VerbPrefixTableEmptyCell{}%
  \VerbPrefixTableEmptyCell{}%
  \VerbPrefixTablePrefixCell{yI-}%
  \VerbPrefixTablePrefixCell{tI-}%
  \VerbPrefixTablePrefixCell{pe-}%
  \VerbPrefixTablePronounCell{tlhIH (ra')}%
\\\hhline{*{8}{|-}|~|}%
  \VerbPrefixTableEmptyCell{|}%
  \VerbPrefixTableEmptyCell{}%
  \VerbPrefixTablePrefixCell{qa-}%
  \VerbPrefixTablePrefixCell{Sa-}%
  \VerbPrefixTablePrefixCell{vI-}%
  \VerbPrefixTablePrefixCell{vI-}%
  \VerbPrefixTablePrefixCell{jI-}%
  \VerbPrefixTablePronounCell{jIH}%
\\\hhline{*{8}{|-}|~|}
  \VerbPrefixTableEmptyCell{|}%
  \VerbPrefixTableEmptyCell{}%
  \VerbPrefixTablePrefixCell{pI-}%
  \VerbPrefixTablePrefixCell{re-}%
  \VerbPrefixTablePrefixCell{wI-}%
  \VerbPrefixTablePrefixCell{DI-}%
  \VerbPrefixTablePrefixCell{ma-}%
  \VerbPrefixTablePronounCell{maH}%
\\\hhline{*{8}{|-}|~|}
  \VerbPrefixTablePrefixCell{cho-}%
  \VerbPrefixTablePrefixCell{ju-}%
  \VerbPrefixTableEmptyCell{}%
  \VerbPrefixTableEmptyCell{}%
  \VerbPrefixTablePrefixCell{Da-}%
  \VerbPrefixTablePrefixCell{Da-}%
  \VerbPrefixTablePrefixCell{bI-}%
  \VerbPrefixTablePronounCell{SoH}%
\\\hhline{*{8}{|-}|~|}
  \VerbPrefixTablePrefixCell{tu-}%
  \VerbPrefixTablePrefixCell{che-}%
  \VerbPrefixTableEmptyCell{}%
  \VerbPrefixTableEmptyCell{}%
  \VerbPrefixTablePrefixCell{bo-}%
  \VerbPrefixTablePrefixCell{bo-}%
  \VerbPrefixTablePrefixCell{Su-}%
  \VerbPrefixTablePronounCell{tlhIH}%
\\\hhline{*{8}{|-}|~|}
  \VerbPrefixTablePrefixCell{mu-}%
  \VerbPrefixTablePrefixCell{nu-}%
  \VerbPrefixTablePrefixCell{Du-}%
  \VerbPrefixTablePrefixCell{lI-}%
  \VerbPrefixTablePrefixCell{-}%
  \VerbPrefixTablePrefixCell{-}%
  \VerbPrefixTablePrefixCell{-}%
  \VerbPrefixTablePronounCell{ghaH/'oH}%
\\\hhline{*{8}{|-}|~|}
  \VerbPrefixTablePrefixCell{mu-}%
  \VerbPrefixTablePrefixCell{nu-}%
  \VerbPrefixTablePrefixCell{nI-}%
  \VerbPrefixTablePrefixCell{lI-}%
  \VerbPrefixTablePrefixCell{lu-}%
  \VerbPrefixTablePrefixCell{-}%
  \VerbPrefixTablePrefixCell{-}%
  \VerbPrefixTablePronounCell{chaH/bIH}%
\\\hhline{*{8}{|-}|~|}
  \VerbPrefixTablePrefixCell{vI-lu'}%
  \VerbPrefixTablePrefixCell{wI-lu'}%
  \VerbPrefixTablePrefixCell{Da-lu'}%
  \VerbPrefixTablePrefixCell{bo-lu'}%
  \VerbPrefixTablePrefixCell{-lu'}%
  \VerbPrefixTablePrefixCell{lu-lu'}%
  \VerbPrefixTablePrefixCell{-lu'}%
  \VerbPrefixTablePronounCell{vay'}%
\\\hline
\end{tabular}}}

% ================================
% Insert A Number of (Empty) Pages
% ================================
% [2000-10-26] Written by Zrajm C Akfohg.
%
% This routine inserts a number of empty pages into a
% document to fill it out. If for example you are intending
% to staple a folder or print a book, you need a specific
% number of pages, e.g. divideble by four. Well, this little
% thingy does just that.
%
% #1 = number of first page in the document (probably `1', but
%      other numbers occur - `3' is quite common)
% #2 = number of pages with print last in the book
%      (i.e. the number of pages *after* the ones being inserted)
% #3 = the number you want to be able to divde the book evenly in
%      (probably something like `4' or `8' for a book)
%
\newcounter{noofpages}
\newcounter{evennoofpages}
\newcommand{\filloutpages}[3]{%
  \setcounter{noofpages}{ \value{page} + #2 - #1 }%
  \setcounter{evennoofpages}{ ( \value{noofpages} / #3 * #3 ) + #3 }%
  \whiledo{\value{noofpages} < \value{evennoofpages}}{%
    ~\clearpage\addtocounter{noofpages}{1}%
  }%
}



% =============================================
% Set a margins and some other useful distances
% =============================================
% [2001-02-28] Written by Zrajm C Akfohg.
%
% #1 is the left margin (on odd/right hand page)
% #2 is the upper margin
% #3 is the right margin (on odd/right hand page)
% #4 is the lower margin
% #5 is the height of header/footer
% #6 is the distance between header/footer and main text
% #7 is the thickness of the ruler header/footer and main text
% #8 is the distance between columns
% #9 is the thickness of the ruler between columns
%
% The margin space for notes is set to zero,
% and header and footer are always set to the same size.
% The «vmargin» package is used to set the margins.
%
% all nine arguments are required (ha!)
\newcommand{\setmargz}[9]{%
  %
  %  The first four arguments of \setmarginsrb are used to set:
  %    \oddsidemargin \evensidemargin \textwidth \topmargin \textheight
  %  The next four sets:
  %    \headheight \headsep \footheight \footskip
  %

  % page margins
  \setpapersize{A4}                       % set papersize for vmargin
  \setlength{\footskip}{#5 + #6}          % correct footskip
  \setmarginsrb{#1}{#2}{#3}{#4}%          % left/upper/right/lower margin
    {#5}{#6}{#5}{\footskip}               % head-y/headsep/foot-y/footskip
  \setlength{\headwidth}{\textwidth}      % header/footer text width

  % rulers and special margins
  %\setlength{\headrulewidth}{#7}          % head ruler thickness
  %\setlength{\footrulewidth}{#7}          % foot ruler thickness
  \setlength{\columnsep}{#8}             % dist. between columns
  \setlength{\columnseprule}{#9}             % dist. between columns
  \setlength{\marginparsep}{0pt}          % del dist. between text & marg note
  \setlength{\marginparwidth}{0pt}        % del width of margin note
  %\setlength{\marginparpush}{0pt}        % min. dist. between margin notes
}






%%%%%%%%%%%%%%%%%%%%%%%%%%%%%%%%%%%%%%%%%%%%%%%%%%%%%%%%%%%%%%%%%%%%%%%%%%%%%%
%%%%%%%%%%%%%%%%%%%%%%%%%%%%%%%%%%%%%%%%%%%%%%%%%%%%%%%%%%%%%%%%%%%%%%%%%%%%%%
%%%                                                                        %%%
%%%     C O D E   T H A T   I S   T O   B E   E R A S E D   B E L O W      %%%
%%%                                                                        %%%
%%%%%%%%%%%%%%%%%%%%%%%%%%%%%%%%%%%%%%%%%%%%%%%%%%%%%%%%%%%%%%%%%%%%%%%%%%%%%%
%%%%%%%%%%%%%%%%%%%%%%%%%%%%%%%%%%%%%%%%%%%%%%%%%%%%%%%%%%%%%%%%%%%%%%%%%%%%%%


% ==========
% Font sizes
% ==========
%\newlength{\headfontsize}                   % Add size for head/foot font 
%\newlength{\headfontheight}                 % Add height for head/foot font
%\newlength{\normalfontsize}                 % Add size for normal text
%\newlength{\normalfontheight}               % Add height for normal text
%\newlength{\smallfontsize}                  % Add size for small text
%\newlength{\smallfontheight}                % Add height for small text
% Make any changes here!!!
%\setlength{\headfontsize}{20pt}             % Set size of head/foot font
%\setlength{\headfontheight}{24pt}           % Set height of head/foot field
%\setlength{\normalfontsize}{16pt}           % Set size of normal text
%\setlength{\normalfontheight}{19.2pt}       % Set row height for nomal text
%\setlength{\smallfontsize}{12pt}            % Set size of small text
%\setlength{\smallfontheight}{14.4pt}        % Set row height for small text

% ===================================================
% Page formatting (margins, header/footer sizes etc.)
% ===================================================


%\newlength{\innermargin}                    % Add inner margin value
%\newlength{\outermargin}                    % Add outer margin value
%\newlength{\uppermargin}                    % Add upper margin value
%\newlength{\lowermargin}                    % Add lower margin value
%\newlength{\headfootsep}                    % Add head/foot separator value
%\newlength{\headfootheight}                 % Add head/foot height value
%\newlength{\headfootline}                   % Add head/foot line thickness
% Make any changes here!!!
%\setlength{\innermargin}{3cm}               % Set inner margin
%\setlength{\outermargin}{2cm}               % Set outer margin
%\setlength{\uppermargin}{1.5cm}             % Set upper margin value
%\setlength{\lowermargin}{1.5cm}             % Set lower margin value
%\setlength{\headfootsep}{5mm}               % Set head/foot separator value
%\setlength{\headfootheight}{\headfontheight}% Set head/foot height value
%\setlength{\headfootline}{1pt}              % Set head/foot line thickness


%\newcommand{\setmargZ}{%
% Calc necessary values for LaTeX (don't change!)
%\setlength{\evensidemargin}{0cm}            % Even (or left) page margins
%\setlength{\oddsidemargin}{\innermargin}    % Odd (or right) page margins
%  \addtolength{\oddsidemargin}{-\outermargin}
%\setlength{\hoffset}{-1in}                  % Horizontal offset
%  \addtolength{\hoffset}{\outermargin}
%\setlength{\textwidth}{\paperwidth}         % Text width
%  \addtolength{\textwidth}{-\innermargin}
%  \addtolength{\textwidth}{-\outermargin}
%\setlength{\headwidth}{\textwidth}          %%% header/footer text width
%\setlength{\voffset}{-1in}                  % Vertical offset
%\setlength{\topmargin}{\uppermargin}
%\setlength{\headheight}{\headfootheight}    % Head height
%\setlength{\headsep}{\headfootsep}          % Head separator
%\setlength{\footskip}{\headfootsep + \headfootheight} %%%
%\setlength{\textheight}{\paperheight}       % Text height
%  \addtolength{\textheight}{-\topmargin}
%  \addtolength{\textheight}{-\headheight}
%  \addtolength{\textheight}{-\headsep}
%  \addtolength{\textheight}{-\footskip}
%  \addtolength{\textheight}{-\lowermargin}
%\setlength{\headrulewidth}{\headfootline}   % Thickness of header line
%\setlength{\footrulewidth}{\headfootline}   % Thickness of footer line
%}


%% A large font for headers & footers
%\newcommand{\headfont}{%
%  \fontsize{\headfontsize}{\headfontheight}%
%  \usefont{OT1}{bch}{m}{n}%         % see LaTeX2e p.54 '{enc}{fam}{ser}{shape}'
%}

%% A small serif font for comments etc.
%\newcommand{\smllfont}{%
%  \fontsize{\smallfontsize}{\smallfontheight}%
%  \usefont{OT1}{bch}{m}{n}%
%}

%% A medium serif font for main text
%\newcommand{\nrmlfont}{%
%  \fontsize{\normalfontsize}{\normalfontheight}%
%  \usefont{OT1}{bch}{m}{n}%
%}

%% A sans serif font for lookup words
%\newcommand{\sansfont}{%
%  \fontsize{\normalfontsize}{\normalfontheight}%
%  \usefont{T1}{pag}{b}{n}%
%}



%%
%% Klingon-English entries
%%
%%   #1 = Klingon (lookup) word
%%   #2 = English definiton
%%   #3 = word type (n, v, adv etc.)
%%   #4 = extra info (regional, slang etc.)
%%   #5 = source (only displayed if not TKD)
%%
%\newcommand{\lukli}[5]{%
%  \hangindent=2.5mm%                       % set hanging indent
%  {\sans\B{#1}}%                           % klingon lookup word (sansserif)
%  \markboth{#1}{#1}%                       % page header info
%  \ \textit{#3}%                           % part of speech
%  \ifthenelse{\equal{#4}{}}%               % regional/slang info (if any)
%    {}{{\small\ #4}}%                      % --||--
%  \ #2%                                    % english definition
%  \ifthenelse{\equal{#5}{TKD}}%            % source (if not TKD)
%    {}{{\small\ [#5]}}%                    % --||--
%}
